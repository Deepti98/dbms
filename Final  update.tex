\documentclass[12pt,a4paper]{article}
\usepackage[utf8]{inputenc}
\usepackage[utf8]{inputenc}
\usepackage{graphicx}
\title{Designing and using a MySQL database for human resource management \\ Final Update}
\date{20 jan 2022}
\author{Deepti Yadav \\ Roll No : 19111023 }
\begin{document}
\maketitle
\section*{Table of Content}
\begin{itemize}
    \item Abstract
\item Introduction
\item definition of Human Resource Information System
\item Development of Human Resource Information Systems
\item introduction to database
\item Data Definition Language (DDL)
\item Data Manipulation Language (DML)
\item Data Control Language (DCL)
\item Database Management System
\item Database Life Cycle 
\item Discussion
\item Conclusion
\item Reference
\end{itemize}
\section*{ABSTRACT : }This paper, we would like to discuss the methods and ideologies used to create a database To be used for Human Resource purposes. Because we want this database to be accessible Anywhere at any time, it is safe to conclude that an internet-based (web-based) database Would work best, therefore we chose MySQL as the database management system. Before Diving deep into the making of the database itself, it is important to understand certain Basics of database design and the phases of the database life cycle itself. 
This paper will Conclude on a few of the basic things to understand before designing a database for Human Resource Management on MySQL
\section*{Keyword : } \textbf{Database, Design Relational Database}
\section*{INTRODUCTION : }Every company in the world has more than one person working For its purpose. These people involved in a company are what we Know as employees, human resources to a company that drives the Company closer to its goal and purpose. As a company grows, it Will need more employees to expand its wings, and to some Human Resource Managers, it is a pain to organize and keep tabs On all the human resource data. The bigger the company, the harder The Human Resource Managers will have to work to keep the data Accurate and tidy.. These Companies also realize the importance of good management of Human resource data by implementing an employee database. Not that doing all the human resource management work Manually is painful and tiring for the Human Resource Managers And administrators, it may also cause the company and owner some Great loss. Imagine having a construction company with hundred Of workers working different shifts and all their attendance data must be recorded manually using paper and pen without Supervision. The workers might put in false data on their Attendance record resulting in a big loss for the company where the Company pays a big amount of money for salaries and wages while The construction project itself has not progressed much. Another problem is that by doing every human resource Process manually, it will take longer for the Human Resource Manager or administrator to work on the employees’ payroll Calculation.All the problems stated above may be solved by having a Database that covers human resource management, from employee Personal data, employee attendance data, to their payroll data.
\section*{definition of Human Resource Information System:-}Human Resource Information System (HRIS) is a software that Keeps all data regarding human resource management activities, From the employee master data, employee attendance data, payroll Data, even their performance (for more advanced measures) in a centralized manner. Improvement in performance levels is very Much possible when using a proper Human Resource Information System, and thus allows consultation activities to take place, such As discussion and coaching or mentoring   Using HRIS also Gives us data that are more accurate and consistent, therefore Resulting in accurate processed data ].
\section*{Development of Human Resource Information Systems : -}The size of the company impacts what gains and losses will be received by the implementer of the HRIS. Larger companies which Already possess technological properties will most definitely feel More benefits as they can retrieve important staff data in a flash.
Small businesses may not have the resources and technical Competence to do such things.A simple and good alternative to creating a customized HRIsIs to use a pre-packaged HRIS software instead. To buy the most Suitable package for a company, it is necessary to contact the Supplier or vendor and compare all of the candidate products to Find the HRIS product that best suits the requirements of the Business.
. One of the most important resources that can be Provided by senior managers are financial resources, which plays A critical role in an HRIS implementation in an organization .Senior managers need to understand exactly why an HRIS in Managing the organization’s human resource activities is Important, and the business values they will gain after implementing an HRIS [12]. By understanding such matters, Senior managers may support the implementation by providing a Bigger budget. This money provided by senior managers will not Only play a role in the implementation, but also operation and Maintenance.
\section*{introduction to database:-} A database is an integrated collection of Data that is stored, managed and centrally controlled. Databases Usually store information about entities in large numbers (tens, Hundreds, to thousands of entities). Information stored in the Database is the entity’s attributes (for example, name, price, and Account balance) and the relationship of an entity with other Entities (for example, which orders belong to which customers). The database has its own language called query. 
There are several kinds of languages in the query where the explanation is as follows:
\begin{itemize}
    \item \textbf{Data Definition Language (DDL) : }which is a language that Governs the making of structures from a database. DDL is Not related to data that fills in the database structure. Some Examples of DDL are CREATE, DROP, and ALTER.
\item \textbf{Data Manipulation Language (DML) : }which is a language That governs changesto database contents. Some examples Of DML are INSERT, UPDATE, DELETE, and SELECT.
\item \textbf{Data Control Language (DCL) : }which a language that Regulates anyone who can access a part of the database Designed. Some examples of DCL are GRANT and REVOKE.
\end{itemize} 
\section*{Database Management System :-}A Database Management System is a Software that allows users to create, access and manage a database.
In the database approach, each file in each department is stored on A database server with a new designation, namely the table. Each Program can then access parts of the database as necessary.
At the moment there are many database management systems Available to meet the business needs of every company. Each Database management system has advantages and disadvantages of Each, where the decision to choose the right candidate is certainly Based on the scale, budget, and needs of the company itself.
\\
\\
\includegraphics[scale=0.5]{Deepti/du1.jpeg}
\\
\\
According to a survey conducted by DB-Engines in May 2019, the Most commonly used database management system is Oracle, MySQL, and Microsoft SQL Server.
In accordance with the needs and business scale of the Organization, MySQL is one of the best candidates to become a Database management system from an employee database that will Be built and implemented. That’s because MySQL is not too Complex and sufficient for the business scale of the organization Which consists of hundreds of employees, unlike Oracle which is Very complex and is more aimed at companies with a much larger Scale. Because the application to be built is also web-based, so a Database that is suitable for use and in accordance with the scale of the company the organization is MySQL. In addition to these Factors, MySQL is very affordable compared to Oracle, where MySQL is a Database Management System (DBMS) that is free to Use and is licensed by the GNU General Public License, while oracle is a paid DBMS (Express version can be accessed free of Charge, but features -The DBMS features are limited and only Recommended for educational and testing use).MySQL is a database management system that is used through The web where MySQL operates on a server .This provides a Good opportunity for writers to develop prototypes related Applications using a web base.
\section*{Database Life Cycle :-}In order to design a database that is usable in the long run, there Are a few steps that should be followed. One of the most general And common guidelines used is the Database Life Cycle. 
The steps Of the Database Life Cycle are as follows.
\section*{ Database Planning : }Database Planning include management activities that enable The Database Life Cycle’s success in effective and efficient Implementation. This step involves defining the mission statement Dan mission objective. The Mission Statement defines the main Purpose of the database application, the purpose of this database Design project, and clarifies the flow of making a database Application so that the design can be carried out effectively and Efficiently.
Examples of mission statements are “The purpose of Creating a database for XYZ Inc. is to facilitate all the activities of XYZ Inc. clients who use their services and also their daily staff So that the business processes carried out can run efficiently with The availability of information that is complete in real-time. “.
\section*{System Definition}System Definition is the process of determining and Elaborating the boundaries and scope of the database system Development project and the main user view of the database system.
What is meant by the user view here is the access rights of A role (such as manager or supervisor) or department (such as marketing or human resources) to data entities in the database toa project can only use one or two because the scope is too large or resources are not available, and time is limited.
\section*{Requirements Collection and Analysis}Requirements Collection and Analysis is the process of gathering the needs of users or companies so that the database system is designed right on target and in accordance with company needs .
These company needs are documented in the requirements specifications. A few approaches that may be used here are the Centralized Approach, View Integration Approach, or both of them combined.
\section*{Database Design}Database Design is basically the designing of the database from the conceptual, logical, and physical model. There are a few approaches to this step which are Bottom-up, Top-down, Inside out, and Mixed .The Bottom-up Approach is an approach where data modeling starts from the properties or attributes, then alarger entity is identified. The bottom-up approach starts from then initial attributes (entities and relationships) .This approach is suitable for simple database designs with  relatively fewer attributes. The Top-down Approach is an approach where data modeling starts from large entities first, then attributes are identified. 
The top-down approach starts with developing a data model that consists of several or many entities and relationships, then identifies low-level entities, relationships, and associations between attributes.
\section*{DBMS Selection:-}DBMS Selection is a stage in the Database System Development Life Cycle where there will be a comparison Between outstanding DBMS products and the selection of DBMS That is deemed most suitable for database design projects. 
The selection of the DBMS must also consider the needs and requirements of company.
\section*{. Application Design}Application design is a process in which the user interface and Features of a database application are described in detail. It is very important to make the application as user-friendly as possible to Ensure the database will be used in the future .
In the Application design process, there are two main processes that are Key to the application design, the transaction design and the user 
Interface design.
\section*{Prototyping:-}prototyping is an activity of making a prototype of the system As a whole. A Prototype itself is a working model or model that Serves to display some features or functionality of the entire Database system that is built.  There are several techniques that can be used to make prototypes, some of which are as follows:
\begin{itemize}    
\item \textbf{Representation: }Representation is a representative of a Database system that as a whole is quite large. 
\item \textbf{Precision: }Precision means that the prototype that is 
Designed must be determined whether it will be made in Detail, or made in outline only to introduce some key 
Features to the end-user.
\item \textbf{Interactivity : }Interactivity refers to the interaction that 
Occurs between the user and the prototype
\end{itemize}
\section*{Implementation}Implementation is the physical realization of database design And software design. At this stage, the realization of the database Is achieved by using Data Definition Language (DDL) from the Selected DBMS. Apart from the database side, the creation of Graphical User Interface (GUI) is also done as a form of Implementation in terms of its application.
\section*{. Data Conversion and Loading}Data Conversion and Loading is an activity to fill tables that Have been made before with real company data . Data 
Conversion and Loading is a mandatory activity so that the Database system can be used operationally by the compay. Conversion here can mean changing the data format from one Format to another so that it can be used in a new database system. 
Company data itself can be retrieved from legacy systems or even Paper-based manual records. Including real data is somewhat Important, even if it is still in the early stages of the process
. Not Having them included from the start may cause the developers to Modify everything they have built just to make it compatible with The data .
\section*{Testing}Testing is an activity to run a database system that is designed With the aim of finding errors or errors that exist. With testing, the Designer prevents errors when the system has been used Operationally which results in hampered the company’s daily Activities. In testing, the designer not only directs attention to the Coding of the system but also the environment and ambiance of the System. 
Before carrying out testing, the person responsible for Testing must understand the quality standards of what the test Intends to do in order to achieve the appropriate testing results.
\section*{Operational Maintenance}Operational Maintenance is the activity of monitoring Activities done within the system in order to maintain the Performance of the system while the business runs, as usual, using A database system that has been developed. In addition to Maintaining the performance of the database system, it is also Necessary to collect the requirements for the next Database System Development Life Cycle.
\section*{Discussion}After analyzing the ongoing business processes, there are Several problems that can be corrected to reduce the losses Experienced by the company. Some of these problems are:
: Payroll costs are inflated because of fraud in recording Attendance using only paper and pens without close Supervision.
\begin{itemize}
    \item The cost of paying overtime wages that are inflated due To fraud in recording overtime duration.
\item The results of work that are not in accordance with the Cost of salaries and overtime wages incurred (the progress Of the project undertaken is not comparable with Statements of overtime and work from employees).
\item HRD employees have difficulty finding data that has long Been due to manual archive storage without the existence Of a database that is a centralized company data Analysis of business processes that run at the company has Exposed the needs of the company which is a database system for The management of the company’s workforce. From the database System to be designed, the expectation of the organization is that The database system can fulfill the following:
\begin{itemize}
    \item The database system can be the actual data storage Regarding employee attendance (with the help of Attendance machines). This means that the database System can read data in the extension format produced by The attendance machine.
\item The database system can help HRD employees in data Searching so that data search can be done quickly and Accurately. This means that the database system must be Able to display the data needed by HRD employees when HRD employees need them.
\item To accelerate the manual recording transition to database System usage, the user interface of the database Application that is designed is expected to be user-friendly So that it is easy for HRD employees to use to input and Pull data output.
Based on the problems and needs of the database system that Has been described above, the author can suggest a solution whichIs making a database system for the management of the Organization’s human resource with the following explanation:
\item  Creating a database for the organization which will be a Centralized data repository for all HRD operational Activities of the organization such as employee personal Data collection, employee attendance records, also the Calculation of salaries received by employees.
\end{itemize}
\item Making a prototype of a web-based application that will Be a bridge for users (HRD employees of the organization) To input data into the database, process
\end{itemize}
\section*{Conclusion}Having an HR database is a necessity in a company, whether it Is a global company or even a company with a smaller scope. Business processes that were running before the database system Implementation brought many losses for the organization. This is Because the manual recording that is easily done by employees Who are absent. Employees who are absent can be easily recorded
As attending. Employees who are not overtime can be recorded as Overtime due to records that use paper and pens. This resulted in a Very large expenditure on the part of employee salaries, which is Not comparable with the progress of the organization This can be Avoided if the organization uses an absent engine and database System that can ensure employee attendance data is accurate to the Real situation.
Without a proper HR database working as an HRIS, It would be a pain for the HR Manager to work on the employee Data manually and produce the paycheck for each employee. Having an HRIS will ensure the data consistency and accuracy,
Thus resulting in accurate payroll calculation. For a more global Approach, the DBMS used for this HRIS will be MySQL since it Runs on a server and is web-based, allowing both interests to fit in, Whether it is using localhost or the internet.
\section*{Reference :-}
\begin{itemize}
    \item 	Sat zinger, Jackson, Bard, System Analysis and Design with the Unified Process, Pearson, 2012.
\item 	Connolly, T., Beg, C., Database System - A Practical Approach to Design, Implementation, and Management, Pearson, 2015.
\end{itemize}
\\
\\
\textbf{***************END OF FINAL TERM PAPER****************}
\end{document}