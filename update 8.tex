\documentclass{article}
\usepackage[utf8]{inputenc}
\usepackage{graphicx}
\graphicspath{{./images}}
\title{ Designing and using a MySQL database for human resource management\\ Update 8}
\date{10 March 2022}
\author{Deepti Yadav\\ Roll No: 19111023 }
\begin{document}
\maketitle
\section*{Database Design : } Database Design is basically the designing of the database from the conceptual, logical, and physical model. There are a few approaches to this step
which are Bottom-up, Top-down, Inside out, and Mixed .The Bottom-up
Approach is an approach where data modeling starts from the properties or
attributes, then alarger entity is identified. The bottom-up approach starts
from then initial attributes (entities and relationships) .This approach is
suitable for simple database designs with relatively fewer attributes. The
Top-down Approach is an approach where data modeling starts from large
entities first, then attributes are identified. The top-down approach starts
with developing a data model that consists of several or many entities and
relationships, then identifies low-level entities, relationships, and associations
between attributes.
\\
\section*{DBMS Selection:- }DBMS Selection is a stage in the Database System Development Life Cycle where there will be a comparison Between outstanding DBMS products
and the selection of DBMS That is deemed most suitable for database design projects. The selection of the DBMS must also consider the needs and
requirements of company.
\\
\section*{Application Design : }Application design is a process in which the user interface and Features of
a database application are described in detail. It is very important to make
the application as user-friendly as possible to Ensure the database will be
used in the future . In the Application design process, there are two main
processes that are Key to the application design, the transaction design and
the user Interface design.
\\
\section*{Prototyping:-}prototyping is an activity of making a prototype of the system As a whole.
A Prototype itself is a working model or model that Serves to display some
features or functionality of the entire Database system that is built. There
are several techniques that can be used to make prototypes, some of which
are as follows:
\begin{itemize}
    \item \textbf{Representation:}Representation is a representative of a Database
system that as a whole is quite large.
\\
\\
\item \textbf{Precision: }: Precision means that the prototype that is Designed must
be determined whether it will be made in Detail, or made in outline
only to introduce some key Features to the end-user.
\\
\\
\item \textbf{Interactivity :} Refers to the interaction that Occurs between the user and the prototype.
\end{itemize}
\end{document}