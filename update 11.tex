\documentclass{article}
\usepackage[utf8]{inputenc}
\usepackage{graphicx}
\graphicspath{{./images}}
\title{ Designing and using a MySQL database for human resource management\\ Update 11}
\date{1 April 2022}
\author{Deepti Yadav\\ Roll No: 19111023 }
\begin{document}
\maketitle
\section*{Conclusion}Having an HR database is a necessity in a company, whether it Is a global
company or even a company with a smaller scope. Business processes that
were running before the database system Implementation brought many
losses for the organization. This is Because the manual recording that is
easily done by employees Who are absent. Employees who are absent can
be easily recorded As attending. Employees who are not overtime can be
recorded as Overtime due to records that use paper and pens. This resulted
in a Very large expenditure on the part of employee salaries, which is Not
comparable with the progress of the organization This can be Avoided if the
organization uses an absent engine and database System that can ensure employee attendance data is accurate to the Real situation. Without a proper
HR database working as an HRIS, It would be a pain for the HR Manager
to work on the employee Data manually and produce the paycheck for each
employee. Having an HRIS will ensure the data consistency and accuracy,
Thus resulting in accurate payroll calculation. For a more global Approach,
the DBMS used for this HRIS will be MySQL since it Runs on a server and
is web-based, allowing both interests to fit in, Whether it is using localhost
or the internet.
\\
\section*{Reference :- }
\begin{itemize}
    \item Sat zinger, Jackson, Bard, System Analysis and Design with the Unified
Process, Pearson, 2012.
\item Connolly, T., Beg, C., Database System - A Practical Approach to
Design, Implementation, and Management, Pearson, 2015.
\end{itemize}

\end{document}