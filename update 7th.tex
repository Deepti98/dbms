\documentclass{article}
\usepackage[utf8]{inputenc}
\usepackage{graphicx}
\graphicspath{{./images}}
\title{ Designing and using a MySQL database for human resource management\\ Update 7}
\date{2 March 2022}
\author{Deepti Yadav\\ Roll No: 19111023 }
\begin{document}
\maketitle
\section*{Database Planning : }Database Planning include management activities that enable The Database
Life Cycle’s success in effective and efficient Implementation. This step involves defining the mission statement Dan mission objective. The Mission
Statement defines the main Purpose of the database application, the purpose
of this database Design project, and clarifies the flow of making a database
Application so that the design can be carried out effectively and Efficiently.
Examples of mission statements are “The purpose of Creating a database for
XYZ Inc. is to facilitate all the activities of XYZ Inc. clients who use their
services and also their daily staff So that the business processes carried out
can run efficiently with The availability of information that is complete in
real-time. “.
\\
\section*{System Definition : }System Definition is the process of determining and Elaborating the boundaries and scope of the database system Development project and the main
user view of the database system. What is meant by the user view here is
the access rights of A role (such as manager or supervisor) or department
(such as marketing or human resources) to data entities in the database toa
project can only use one or two because the scope is too large or resources
are not available, and time is limited.
\\
\section*{Requirements Collection and Analysis : }Requirements Collection and Analysis is the process of gathering the needs of
users or companies so that the database system is designed right on target and
in accordance with company needs . These company needs are documented
in the requirements specifications. A few approaches that may be used here
are the Centralized Approach, View Integration Approach, or both of them
combined.

\end{document}